\chapter{緒論}  %可更改{}中的文字來更改大標題名稱
\label{c:intro}


%若想新增新的節可用\section{}來加入
%段落結束後可使用\par來結束段落
\section{插入公式}

狗狗可愛貓貓更可愛,狗狗可愛貓貓更可愛,狗狗可愛貓貓更可愛,狗狗可愛貓貓更可愛,狗狗可愛貓貓更可愛,狗狗可愛貓貓更可愛,狗狗可愛貓貓更可愛,狗狗可愛貓貓更可愛,狗狗可愛貓貓更可愛,狗狗可愛貓貓更可愛,狗狗可愛貓貓更可愛,狗狗可愛貓貓更可愛,狗狗可愛貓貓更可愛,狗狗可愛貓貓更可愛,狗狗可愛貓貓更可愛,狗狗可愛貓貓更可愛,狗狗可愛貓貓更可愛,狗狗可愛貓貓更可愛,狗狗可愛貓貓更可愛,狗狗可愛貓貓更可愛,狗狗可愛貓貓更可愛,狗狗可愛貓貓更可愛,狗狗可愛貓貓更可愛,狗狗可愛貓貓更可愛,狗狗可愛貓貓更可愛,狗狗可愛貓貓更可愛,狗狗可愛貓貓更可愛,狗狗可愛貓貓更可愛,狗狗可愛貓貓更可愛,狗狗可愛貓貓更可愛,如下式$1.1$。\par

%%可以使用以下格式在文章中插入公式,所插入的公式會自行依照章節編號,正文中引用請使用$X.X$來引用,範例如上

\begin{equation}
h_{t} = c + \sum_{i=1}^{q}\alpha_{i} \epsilon^2_{t-1} + \sum_{j=1}^{p}\beta_{j} h_{t-1} \quad; \mbox{where} \quad v_{t}\overset{\mathrm{iid}}{\sim}(0,1)
\end{equation}



狗狗可愛貓貓更可愛,狗狗可愛貓貓更可愛,狗狗可愛貓貓更可愛,狗狗可愛貓貓更可愛,狗狗可愛貓貓更可愛,狗狗可愛貓貓更可愛,狗狗可愛貓貓更可愛,狗狗可愛貓貓更可愛,狗狗可愛貓貓更可愛,狗狗可愛貓貓更可愛,狗狗可愛貓貓更可愛,狗狗可愛貓貓更可愛,狗狗可愛貓貓更可愛,狗狗可愛貓貓更可愛,狗狗可愛貓貓更可愛,狗狗可愛貓貓更可愛,狗狗可愛貓貓更可愛,狗狗可愛貓貓更可愛,狗狗可愛貓貓更可愛,狗狗可愛貓貓更可愛,狗狗可愛貓貓更可愛,狗狗可愛貓貓更可愛,狗狗可愛貓貓更可愛,狗狗可愛貓貓更可愛,狗狗可愛貓貓更可愛,狗狗可愛貓貓更可愛,狗狗可愛貓貓更可愛。

\section{研究方法}

狗狗可愛貓貓更可愛,狗狗可愛貓貓更可愛,狗狗可愛貓貓更可愛,狗狗可愛貓貓更可愛,狗狗可愛貓貓更可愛,狗狗可愛貓貓更可愛,狗狗可愛貓貓更可愛,狗狗可愛貓貓更可愛,狗狗可愛貓貓更可愛,狗狗可愛貓貓更可愛,狗狗可愛貓貓更可愛,狗狗可愛貓貓更可愛,狗狗可愛貓貓更可愛,狗狗可愛貓貓更可愛,狗狗可愛貓貓更可愛,狗狗可愛貓貓更可愛,狗狗可愛貓貓更可愛,狗狗可愛貓貓更可愛,狗狗可愛貓貓更可愛,狗狗可愛貓貓更可愛,狗狗可愛貓貓更可愛,狗狗可愛貓貓更可愛,狗狗可愛貓貓更可愛,狗狗可愛貓貓更可愛,狗狗可愛貓貓更可愛,狗狗可愛貓貓更可愛,狗狗可愛貓貓更可愛。